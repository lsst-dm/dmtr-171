\documentclass[DM,lsstdraft,STR,toc]{lsstdoc}
\usepackage{geometry}
\usepackage{longtable,booktabs}
\usepackage{enumitem}
\usepackage{arydshln}

\input meta.tex

\providecommand{\tightlist}{
  \setlength{\itemsep}{0pt}\setlength{\parskip}{0pt}}

\begin{document}

\def\milestoneName{ComCam interface verification readiness}
\def\milestoneId{LDM-503-6}
\def\product{Data Management}

\setDocCompact{true}

\title{LDM-503-6: ComCam Interface Verification Readiness Test Plan and Report}
\setDocRef{\lsstDocType-\lsstDocNum}
\date{\vcsdate}
\setDocUpstreamLocation{\url{https://github.com/lsst/lsst-texmf/examples}}
\author{ Michelle Butler }

\input history_and_info.tex


\setDocAbstract{
This is the test plan and report for LDM-503-6 (ComCam interface verification readiness), an LSST level 2 milestone pertaining to the Data Management Subsystem.
}


\maketitle

\section{Introduction}
\label{sect:intro}


\subsection{Objectives}
\label{sect:objectives}

This test will verify the readiness of the ComCam DM interface. This
milestone records successful transfer of an image equivalent to one raft
from the DAQ at the summit to reliable storage in the LSST Data Facility
at NCSA, from where it will be made available for scientific evaluation
through the LSST Science Platform ~There are many steps along the way
from the DAQ to the L1 handoff machine, ~transferred to NCSA and
ingested into the permanent record of the survey. ~~



\subsection{System Overview}
\label{sect:systemoverview}

The system includes the comcam DAQ at the summit, and the comcam DAQ
network connecting the Comcam systems to the support infrastructure at
the base. ~ On the comcam DAQ network is the CC-header service system,
the CC-forwarder, CC-archiver, and on a base network the L1 handoff with
the OODS and DBB services. ~At NCSA, the DBB ingest environment, the raw
image file systems, the ingested image file systems and the LSP notebook
environment. ~ This milestone includes many steps. ~ It will ensure that
image data can be acquired from the camera, ~a single YAML header file
is generated for all 9 CCDs (or sensors), is ingested into a butler into
BDC file systems, served through NFS for the base systems for quick peek
capabilities, made available for DBB resources, transferred to NCSA,
ingested by a butler into ~NCSA filesystems and made available for
processing/viewing by the LSP. ~\\[2\baselineskip]

\subsection{Applicable Documents}\label{applicable-documents}

\citeds{LDM-294} Data Management Organization and Management\\
\citeds{LDM-503} DM Test Plan\\
\citeds{LDM-148} Data Management System Design\\[2\baselineskip]


\subsection{Document Overview}
\label{sect:docoverview}

This document was generated from Jira, obtaining the relevant information from the 
\href{https://jira.lsstcorp.org/secure/Tests.jspa#/testPlan/LVV-P50}{LVV-P50}
~Jira Test Plan and related Test Cycles (
  \href{https://jira.lsstcorp.org/secure/Tests.jspa#/testCycle/LVV-C106}{LVV-C106}
).

Section \ref{sect:intro} provides an overview of the test campaign, the system under test (\product{}), the applicable documentation, and explains how this document is organized.
Section \ref{sect:configuration}  describes the configuration used for this test.
Section \ref{sect:personnel} describes the necessary roles and lists the individuals assigned to them.
%Section \ref{sect:plannedtestactivities} provides the list of planned test cycles and test cases, including all relevant information that fully describes the test campaign.

Section \ref{sect:overview} provides a summary of the test results, including an overview in Table \ref{table:summary}, an overall assessment statement and suggestions for possible improvements.
Section \ref{sect:detailedtestresults} provides detailed results for each step in each test case.

The current status of test plan LVV-P50 in Jira is \textbf{ Approved }.

\subsection{References}
\label{sect:references}
\renewcommand{\refname}{}
\bibliography{lsst,refs,books,refs_ads}
\section{Test Configuration}
\label{sect:configuration}

\subsection{Data Collection}

  Observing is not required for this test campaign.

\subsection{Verification Environment}
\label{sect:hwconf}
  The verification environment includes the DAQ, cc-archiver,
cc-header-service, CC-forwarder, L1-handoff, long haul network, NCSA DBB
gateway environments, GPFS file systems at NCSA, and the LSP at NCSA. ~~


  \subsection{Entry Criteria}
  Images are being taken from the com-cam environment from the summit with
unique names and complete headers.~


  \subsection{Exit Criteria}
  Images are being ingested and available on the LSP at the LDF for
viewing and processing by science staff at NCSA.~~



\newpage
\section{Personnel}
\label{sect:personnel}

The following personnel are involved in this test activity:

\begin{itemize}
\item Test Plan (LVV-P50) owner: Michelle Butler
\item Test Cycles:
\begin{itemize}
  \item LVV-C106 owner: 
    Michelle Butler
  \begin{itemize}
    \item Test case \href{https://jira.lsstcorp.org/secure/Tests.jspa#/testCase/LVV-T1549}{LVV-T1549} tester: 
  \end{itemize}
\end{itemize}
\item Additional Test Personnel involved:
  \begin{itemize}
    \item Test case \href{https://jira.lsstcorp.org/secure/Tests.jspa#/testCase/LVV-T1549}{LVV-T1549}: Michelle Gower and Michelle Butler ~and Steve P and Felipe M.~

  \end{itemize}
\end{itemize}

\newpage

\section{Overview of the Test Results}
\label{sect:overview}

\subsection{Summary}
\label{sect:summarytable}

\begin{longtable}{p{0.12\textwidth}p{0.2\textwidth}p{0.56\textwidth}p{0.12\textwidth}}
\toprule

  \multicolumn{3}{c}{ Test Cycle {\bf LVV-C106: LDM-503-6 Comcam verification readiness
 }} \\\hline

  {\bf \footnotesize test case} & {\bf \footnotesize status} & {\bf \footnotesize comment} & {\bf \footnotesize issues} \\\toprule

    \href{https://jira.lsstcorp.org/secure/Tests.jspa#/testCase/LVV-T1549}{LVV-T1549}
    & Not Executed &  &
    \\\hline

\caption{Test Results Summary}
\label{table:summary}
\end{longtable}

\subsection{Overall Assessment}
\label{sect:overallassessment}

Not yet available.

\subsection{Recommended Improvements}
\label{sect:recommendations}

Not yet available.

\newpage
\section{Detailed Test Results}
\label{sect:detailedtestresults}


  \subsection{Test Cycle LVV-C106 }

Open test cycle {\it \href{https://jira.lsstcorp.org/secure/Tests.jspa#/testrun/LVV-C106}{LDM-503-6 Comcam verification readiness
}} in Jira.

  LDM-503-6 Comcam verification readiness
\\
  Status: Not Executed

  Verify that the DM services required for ComCam exist and are ready to
recieve a image from the comcam DAQ at the summit and transfer the file
to NCSA for viewing on the LSP.~ ~


  \subsubsection{Software Version/Baseline}
    Not provided.

  \subsubsection{Configuration}
    There are many pieces here. ~ ~ Comcam DAQ, Header service, comcam
archiver/forwarder, AT-handoff, DTN transfers, ~DM butler/G3, LDF file
systems, LDF LSP.~


  \subsubsection{Test Cases in LVV-C106 Test Cycle}


    \paragraph{Test Case LVV-T1549 - LDM-503-6 Comcam verification readiness
 }\mbox{}\\

Open  \href{https://jira.lsstcorp.org/secure/Tests.jspa#/testCase/LVV-T1549}{\textit{ LVV-T1549 } }
test case in Jira.

    Verify that Comcam has all the services running and verified working for
retrieving an image from the comcam DAQ and store it on file systems at
the LDF for viewing by LSP. ~


    \textbf{ Preconditions}:\\
    Comcam must be at the summit and producing images with proper headers. ~


    Execution status: {\bf Not Executed }

    Final comment:\\


    Detailed step results:

    \begin{longtable}{p{1cm}p{2cm}p{13cm}}
    \hline
    {Step} & \multicolumn{2}{c}{Description, Results and Status}\\ \hline
      1 & Description &

      \begin{minipage}[t]{13cm}{\footnotesize
      CC-DAQ produces an image~

      \vspace{\dp0}
      } \end{minipage} \\
      \\ \cdashline{2-3}


      & Expected Result &

      \begin{minipage}[t]{13cm}{\footnotesize
      in memory file created in DAQ

      \vspace{\dp0}
      } \end{minipage} \\
      \\ \cdashline{2-3}

      & \begin{minipage}[t]{2cm}{Actual\\ Result}\end{minipage}   & 
      \begin{minipage}[t]{13cm}{\footnotesize
      
      \vspace{\dp0}
      } \end{minipage} \\
      \\ \cdashline{2-3}


      & Status          & Not Executed \\ \hline

      2 & Description &

      \begin{minipage}[t]{13cm}{\footnotesize
      Archiver and Forwarder build image with proper header from header
service~

      \vspace{\dp0}
      } \end{minipage} \\
      \\ \cdashline{2-3}


      & Expected Result &

      \begin{minipage}[t]{13cm}{\footnotesize
      9 image files all with individual headers and then 1 header for all 9
images too. ~\\[2\baselineskip]

      \vspace{\dp0}
      } \end{minipage} \\
      \\ \cdashline{2-3}

      & \begin{minipage}[t]{2cm}{Actual\\ Result}\end{minipage}   & 
      \begin{minipage}[t]{13cm}{\footnotesize
      
      \vspace{\dp0}
      } \end{minipage} \\
      \\ \cdashline{2-3}


      & Status          & Not Executed \\ \hline

      3 & Description &

      \begin{minipage}[t]{13cm}{\footnotesize
      AT-archiver/forwarder transfers the file to the l1-handoff machine.~

      \vspace{\dp0}
      } \end{minipage} \\
      \\ \cdashline{2-3}


      & Expected Result &

      \begin{minipage}[t]{13cm}{\footnotesize
      image file now found on disk on L1-handoff with hardlinks to 2 different
file systems (OODS and DBB) services. ~\\[3\baselineskip]

      \vspace{\dp0}
      } \end{minipage} \\
      \\ \cdashline{2-3}

      & \begin{minipage}[t]{2cm}{Actual\\ Result}\end{minipage}   & 
      \begin{minipage}[t]{13cm}{\footnotesize
      
      \vspace{\dp0}
      } \end{minipage} \\
      \\ \cdashline{2-3}


      & Status          & Not Executed \\ \hline

      4 & Description &

      \begin{minipage}[t]{13cm}{\footnotesize
      OODS service running on L1-handoff machine ingests the image file into
Butler/G3 and readies the file systems for the commissioning cluster at
the Base to be able to mount and see the new files.~ ~~

      \vspace{\dp0}
      } \end{minipage} \\
      \\ \cdashline{2-3}


      & Expected Result &

      \begin{minipage}[t]{13cm}{\footnotesize
      Image file ingested\\[2\baselineskip]

      \vspace{\dp0}
      } \end{minipage} \\
      \\ \cdashline{2-3}

      & \begin{minipage}[t]{2cm}{Actual\\ Result}\end{minipage}   & 
      \begin{minipage}[t]{13cm}{\footnotesize
      
      \vspace{\dp0}
      } \end{minipage} \\
      \\ \cdashline{2-3}


      & Status          & Not Executed \\ \hline

      5 & Description &

      \begin{minipage}[t]{13cm}{\footnotesize
      DBB transfers the file to NCSA thorough the DBB-gateway machines and DTN
nodes at the base.~~

      \vspace{\dp0}
      } \end{minipage} \\
      \\ \cdashline{2-3}


      & Expected Result &

      \begin{minipage}[t]{13cm}{\footnotesize
      data file arrives at file systems at NCSA\\[2\baselineskip]

      \vspace{\dp0}
      } \end{minipage} \\
      \\ \cdashline{2-3}

      & \begin{minipage}[t]{2cm}{Actual\\ Result}\end{minipage}   & 
      \begin{minipage}[t]{13cm}{\footnotesize
      
      \vspace{\dp0}
      } \end{minipage} \\
      \\ \cdashline{2-3}


      & Status          & Not Executed \\ \hline

      6 & Description &

      \begin{minipage}[t]{13cm}{\footnotesize
      Files are ingested into the butler/G3 at NCSA and moved to file systems
that are viewable by the LSP. ~

      \vspace{\dp0}
      } \end{minipage} \\
      \\ \cdashline{2-3}


      & Expected Result &

      \begin{minipage}[t]{13cm}{\footnotesize
      data can be seen and retrieved by LSP.~~

      \vspace{\dp0}
      } \end{minipage} \\
      \\ \cdashline{2-3}

      & \begin{minipage}[t]{2cm}{Actual\\ Result}\end{minipage}   & 
      \begin{minipage}[t]{13cm}{\footnotesize
      
      \vspace{\dp0}
      } \end{minipage} \\
      \\ \cdashline{2-3}


      & Status          & Not Executed \\ \hline

    \end{longtable}


\newpage
\appendix
%Make sure lsst-texmf/bin/generateAcronyms.py is in your path
\section{Acronyms used in this document}\label{sec:acronyms}
\addtocounter{table}{-1}
\begin{longtable}{|p{0.145\textwidth}|p{0.8\textwidth}|}\hline
\textbf{Acronym} & \textbf{Description}  \\\hline

CC & Change Control \\\hline
DAQ & Data Acquisition System \\\hline
DBB & Data Back Bone \\\hline
DM & Data Management \\\hline
DMTN & DM Technical Note \\\hline
DTN & Data Transfer Node \\\hline
EFD & Engineering and Facility Database \\\hline
GPFS & General Parallel File System (now IBM Spectrum Scale) \\\hline
LDM & LSST Data Management (Document Handle) \\\hline
LSP & LSST Science Platform \\\hline
LSST & Large Synoptic Survey Telescope \\\hline
NCSA & National Center for Supercomputing Applications \\\hline
NFS & Network File System \\\hline
OODS & Observatory Operations Data Service \\\hline
Subsystem & A set of elements comprising a system within the larger LSST system that is responsible for a key technical deliverable of the project \\\hline
YAML & Yet Another Markup Language \\\hline
\end{longtable}


\end{document}
